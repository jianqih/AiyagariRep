\documentclass[11pt,a4paper]{article}

\usepackage{graphicx,hyperref,amsmath,natbib,bm,url}
\usepackage[utf8]{inputenc} % include umlaute

% -------- Define new color ------------------------------
\usepackage{color}
\definecolor{mygrey}{rgb}{0.85,0.85,0.85} % make grey for the table
%   }}
\definecolor{darkblue}{rgb}{0.055,0.094,0.7}
\definecolor{darkred}{rgb}{0.6,0,0}

% --------------------------------------------------------
\usepackage{microtype,todonotes}
\usepackage[australian]{babel} % change date to to European format
\usepackage[a4paper,text={14.5cm, 25.2cm},centering]{geometry} % Change page size
\setlength{\parskip}{1.2ex} % show new paragraphs with a space between lines
\setlength{\parindent}{0em} % get rid of indentation for new paragraph
\clubpenalty = 10000 % prevent "orphans" 
\widowpenalty = 10000 % prevent "widows"

\hypersetup{pdfpagemode=UseNone} % un-comment this if you don't want to show bookmarks open when opening the pdf

\usepackage{mathpazo} % change font to something close to Palatino

\usepackage{longtable, booktabs, tabularx} % for nice tables
\usepackage{caption,fixltx2e}  % for nice tables
\usepackage[flushleft]{threeparttable}  % for nice tables

\newcommand*{\myalign}[2]{\multicolumn{1}{#1}{#2}} % define new command so that we can change alignment by hand in rows
 
% -------- For use in the bibliography -------------------
\usepackage{multicol}
\usepackage{etoolbox}
\usepackage{relsize}
\setlength{\columnsep}{1cm} % change column separation for multi-columns
% \patchcmd{\thebibliography}
%   {\list}
%   {\begin{multicols}{2}\smaller\list}
%   {}
%   {}
% \appto{\endthebibliography}{\end{multicols}}
% % --------------------------------------------------------


\usepackage[onehalfspacing]{setspace} % Line spacing

\usepackage[marginal]{footmisc} % footnotes not indented
\setlength{\footnotemargin}{0.2cm} % set margin for footnotes (so that the number doesn't stick out) 

\usepackage{pdflscape} % for landscape figures

\usepackage[capposition=top]{floatrow} % For notes below figure

% -------- Use the following two lines for making lines grey in tables -------------------
\usepackage{color, colortbl}
\usepackage{multirow}
% ----------------------------------------------------------------------------------------
\newtheorem{defn}{Definition}[section]
\newtheorem{reg}{Rule}[section]
\newtheorem{exer}{Exercise}[section]
\newtheorem{note}{Note}[section]
\newtheorem{expl}{Example}[section]
\newtheorem{pro}{properties}[section]
\newtheorem{asm}{Assumption}
\newtheorem{them}{Theorem}[section]
\newtheorem{rmk}{Remark}[section]
\newtheorem{coro}{Corollary}[section]
\newtheorem{lem}{Lemma}[section]
\newtheorem{pros}{Proposition}[section]


\usepackage{dsfont}
% \tcbset{highlight math style={enhanced,
%     colframe=red!60!black,colback=yellow!50!white,arc=4pt,boxrule=1pt,
\hypersetup{colorlinks=true,           % put a box around links
  linkbordercolor = {1 0 0}, % the box will be red
  pdfborder = {1 0 0},       % 
  % bookmarks=true,            % PDF will contain an index on the RHS
  urlcolor=darkred,
  citecolor=darkblue,
  linkcolor=darkred
}
% The following defines a footnote without a marker
\newcommand\blfootnote[1]{%
  \begingroup
  \renewcommand\thefootnote{}\footnote{#1}%
  \addtocounter{footnote}{-1}%
  \endgroup
}

% -------- Customize page numbering ----------------------------
\usepackage{fancyhdr}
\usepackage{lastpage}
 \pagestyle{fancy}
\fancyhf{} % get rid of header and footer
 \cfoot{ \footnotesize{ \thepage } }

% \rhead{\small{\textit{Jianqi Huang}}}
% \lhead{\small{\textit{Heterogeneous Agents'n OLG model}}}
\renewcommand{\headrulewidth}{0pt} % turn off header line
% --------------------------------------------------------------

\begin{document}
\paragraph{Model\citep{Aiyagari1994}}
\begin{itemize}
    \item Preference: $\mathbb{E}_0\sum_{i=0}^\infty \beta^t u(C_t)$ 
    \item Budget constraint: $c_t + a_{t+1} = (1+r)a_t + w_t l_t$ where $l_t$ follows the AR(1) process: $l_t = \rho l_{t-1}+\sigma\epsilon_t$ and $\epsilon_t \sim N(0,1)$. 
    \item $a_t \geq -B$ and $c_t\geq 0$
    \item Bellman equation is $v(a,s) = \max_{a_t\geq -B,c_t\geq 0 } u((1+r_t)a_t+w_t l_t-a_{t+1})+ \beta \mathbb{E}[v(a',s')\mid s]$.
    \item C-D production function: $Y_t = A_t K^\alpha_t L^{1-\alpha}_t$ which could drive the prices from FOCs $r = \alpha A_t\left(\frac{K}{L}\right)^{\alpha-1}$ and $w = (1-\alpha )A_t \left(\frac{L}{K}\right)^{\alpha }$. 
    \item Market clearing conditions: $Y = \int c d\mu + K' -(1-\delta )K$ and $L = \int l d\mu$ and capital $K = \int k d\mu$. 
    \item Calibration: set $\beta = 0.96,\gamma=2,B=0,\delta = 0.1,\alpha=0.33$. 
\end{itemize}

\paragraph{Tauchen method} Here we use the Tauchen method to discretize the AR(1) process $$ \log l_t = \rho \log l_{t-1}+\sigma_t \varepsilon_t $$
here $\sigma_z = \frac{1}{\sqrt{1-\rho^2}}$. First, we set $n$, the number of realizations (usually 5 or 6). Second, we pick $m$ and set $z_{\max} = \sigma_z m$ and $z_{\min} = -\sigma_z m$ where usual values of $m$ 2 or 3. Hence, we could write down the discretization $z_i=z_{\min} + \frac{z_{\max}-z_{\min}}{n-1}(i-1)$ for $i=1,\ldots,n$ for equal distance and construct the midpoint $\{\tilde{z}\}_{i=1}^{n-1}$, which are given by $\tilde{z}_i = \frac{z_{i+1}+z_i}{2}$. Lastly, we calculate the transition probabilities $\pi_{z,z'}$ by the following formula: 
\begin{equation}
    \begin{aligned}
        \pi_{ij} &= \Phi \left(\frac{\tilde{z}_j-\rho z_i}{\sigma}\right)- \Phi \left(\frac{\tilde{z}_{j-1}-\rho z_i}{\sigma}\right) \quad j=2,3,\cdots,n-1\\ 
        \pi_{i1} &= \Phi \left(\frac{\tilde{z}_1-\rho z_i}{\sigma}\right)\\
        \pi_{in} &= 1 - \Phi \left(\frac{\tilde{z}_{n-1}-\rho z_i}{\sigma}\right)
    \end{aligned}
\end{equation}
where $\Phi(\cdot)$ denotes a CDF of a $\mathcal{N}(0,1)$. 



\paragraph{Endogenous Grid Method\citep{carrollMethodEndogenousGridpoints2006}} Start from at the end of period level of capital, using the Euler equation, one may recover the beginning-of-period consumption and level of capital without using a root-finding algorithm. 

Cash-on-hand is $m_t = (1+r_t)a_t + w_t l_t$ and the Euler equation is $u'(c_t) = \beta (1+r_{t+1})u'(c_{t+1})$.


Consider the Euler equation $$ u'(c(M)) = \beta R \mathbb{E}_y u'(c(RA+\tilde{y})) $$
if policy function $c(M)$ is optimal, then it satisfies the above equation with $A = M-c(M)$. given any policy function $c(M)$ an updated policy function $c'(M')$ with \textit{parametrized curve} 
$$ \begin{cases}
    c' = (u')^{-1}\left(\beta R \mathbb{E}_y u'(c(RA+\tilde{y}))\right)\\ 
    M' = A + c'
\end{cases} $$
where $A$ is the parameter ranges from $0$ to $M$. 

Recall Coleman-Reffet operator $K(c)(M):\mathcal{P}\to \mathcal{P}$ is defined as 
\begin{itemize}
    \item takes as input policy function $c(M)\in \mathcal{P}$
    \item returns the updated policy function $c'(M)\in \mathcal{P}$ that for every $M$ satisfies the Euler equation. $u'(c'(M)) = \beta R \mathbb{E}_y u'(c[R(M-c'(M))+\tilde{y}])$
\end{itemize}
Standard implementation: fix grid over $M$; with given $c(M)$ solve the equation for $c$ in each point $M$ on the grid. 
EGM implementation of Coleman-Reffet operator:
\begin{enumerate}
    \item Fix grid over $A$;
    \item With given $c(M)$ for each point on the grid compute $$ c' = (u')^{-1}\left(\beta R\mathbb{E}_y u'(c(RA+\tilde{y}))\right)\quad M' = A + c'$$ 
    \item Build the return policy function as interpolation over $(M',c)$ 
\end{enumerate}
$$ M\to c(M)\to A = M-c(M) \to M' = R(M-c(M))+\tilde{y} = RA+\tilde{y} $$

A contains all the information about calculation of $M'$ and $c'$, and $M'$ contains all the information about the calculation of $c'$.

\paragraph{BKM\citep{boppartExploitingMITShocks2018}}




\paragraph{SSJ\citep{auclertUsingSequenceSpaceJacobian2021}}

\bibliographystyle{aer}
\bibliography{ref}

\end{document}